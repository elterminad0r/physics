\documentclass{exam}
\begin{document}
\printanswers
\section{Lattices}
\begin{questions}
\question How might the structure of a normal metal differ from a regular lattice?
\begin{solution}[.2in]
    Metals are likely to have lattice defects such as dislocations.
\end{solution}
\end{questions}
\section{Properties}
\begin{questions}
\question Why do you think dislocations make metals weaker than perfect crystals?
\begin{solution}[.2in]
    Dislocations allow layers of atoms to slip easily, requiring only some
    bonds to change, which means metals are often more easy to deform (more
    malleable).
\end{solution}
\question As you work a metal, it becomes more brittle and less ductile, why?
\begin{solution}[.2in]
    As slipping happens, dislocations in the lattice are eliminated, meaning
    there is less scope for slipping and the metal becomes less ductile.
\end{solution}
\end{questions}
\section{Alloys}
\begin{questions}
\question Why might dislocations be ddisrupted in alloys? Use the diagram and
          your knowledge of dislocations to explain this.
\begin{solution}[.2in]
    Alloy atoms ``pin'' dislocations, by sitting in the gap. This prevents the
    alloy from slipping, so it becomes stronger.
\end{solution}
\end{questions}
\end{document}
