\documentclass[a4paper,11pt]{article}
\usepackage[separate-uncertainty]{siunitx}
\usepackage[siunitx]{gnuplottex}
\usepackage{csvsimple}
\usepackage{booktabs}
\usepackage[margin=1in,tmargin=1in]{geometry}
\usepackage[square,numbers]{natbib}
\usepackage{parskip}
\usepackage{url}
\usepackage{fancyhdr}

\pagestyle{fancy}
\fancyhf{}
\lhead{\url{https://github.com/elterminad0r/physics/tree/master/pressurelaw}}

\begin{document}

\begin{minipage}[c]{0.2\textwidth}
\csvreader[tabular=r r,
    table head=\toprule\textbf{P/\si{\kilo\pascal}} &
               \textbf{T/\si{\degreeCelsius}} \\\midrule,
    table foot=\bottomrule]
{data.csv}{}
{\csvcoli&\csvcolii}
\end{minipage}
\begin{minipage}[c]{0.4\textwidth}
\begin{gnuplot}[terminal=epslatex]
set key autotitle columnhead
unset key
set datafile separator ','
ATMOSPHERE = 101.6
f(x) = m * (x - T_0)
FIT_LIMIT = 1e-4
fit f(x) 'data.csv' u 2:($1+ATMOSPHERE) via m, T_0
set ylabel 'P/\si{\kilo\pascal}'
set xlabel 'T/\si{\degreeCelsius}'
set label 1 sprintf('\SI{0}{\kelvin} = \SI{%.1f \pm %.1f}{\degreeCelsius}',\
            T_0, T_0_err) at -300,100
set yrange [0:150]
set xrange [-400:100]
set title 'Using the pressure law to determine absolute 0'
plot 'data.csv' u 2:($1+ATMOSPHERE), f(x)
\end{gnuplot}
\end{minipage}

The expected value of \SI{0}{\kelvin} being \SI{-273.15}{\degreeCelsius} \cite{SI}.
This lies well outside of the graphically deduced uncertainty. This is an error
of about 30\%. This is probably due to a number of unquantified systematic
errors.

One factor could be the fact that temperature throughout the air sample wasn't
constant, so our reading of temperature may not reflect the actual average
value.

Another contributing factor might be a pressure leak - this makes sense as it
would decrease the gradient, which, at this level of extrapolation can have
significant effects on the $x$-intercept.

\bibliographystyle{agsm}
\bibliography{sources}

\end{document}
