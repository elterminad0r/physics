\documentclass{article}
\title{Formula sheet for testing materials}
\author{Izaak van Dongen}
\usepackage{amsmath}
\usepackage{amsfonts}
\begin{document}
    \maketitle
    \tableofcontents
    \section{Material properties}
        \subsection{Strength}
            Strength is actually a stress value. Strength is the stress at
            which a material "fails".
            \begin{itemize}
                \item
                    The "yield strength" of a material is when it starts to
                    plastically deform.
                \item
                    The "breaking strength" of a material is when it actually
                    breaks.
            \end{itemize}
        \subsection{Young's Modulus}
            Young's modulus is an indication of the stiffness of a material
            (how much strain is produced by a stress while deforming
            elastically). It is given by stress over strain, and denoted as
            $E$:
            \begin{equation}
                E = \frac{\sigma}{\epsilon}
            \end{equation}
            As the units of $\sigma$ are $Pa$ and $\epsilon$ is dimensionless,
            we can conclude that $E$ has units $Pa$.

            Also, as we know formulas for both $\sigma$ and $\epsilon$, we can
            substitute in:
            \begin{align*}
                E &= \frac{\sigma}{\epsilon}\\
                  &= \frac{\frac{F}{A}}{\frac{x}{l}}\\
                  &= \frac{Fl}{Ax}
            \end{align*}
    \section{Formulas}
        \subsection{Stress}
            The stress is defined as force applied divided by cross-sectional
            area and denoted as $\sigma$:
            \begin{equation}
                \sigma = \frac{F}{A}
            \end{equation}
            Hence, stress is measured in $Nm^{-2}$, which is called $Pa$ (scals).
        \subsection{Strain}
            Strain is a measure of the deformation of a material. It denoted as
            $\epsilon$ and given by extension over original length:
            \begin{equation}
                \epsilon = \frac{x}{l_0}
            \end{equation}
            As the units of both $x$ and $l_0$ are $m$,
            \begin{align*}
                \mathrm{units} &= \frac{m}{m}\\
                               &= \mathrm{dimensionless}
            \end{align*}
        \subsection{Area of a circle}
            This is given by
            \begin{equation}
                A_{circle} = \pi r^2
            \end{equation}
            We can also make some substitutions, as we know
            \begin{align*}
                r &= \frac{d}{2}\\
                \Rightarrow A_{circle} &= \frac{\pi d^2}{4}
            \end{align*}
\end{document}
